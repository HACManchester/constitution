\title{Articles of Association of London Hackspace Ltd}
\begin{enumerate}
\section{Interpretation}
\item
  In the articles:
  \begin{description}
  \item[the association] the above-named association.
  \item[Address] means a postal address or, for the purposes of
    electronic communication, a fax number, an e-mail or postal address
    or a telephone number for receiving text messages in each case
    registered with the association
  \item[The Articles] means these articles of association
  \item[Clear days] in relation to the period of a notice means a period
    excluding:
    \begin{itemize}
    \item
      the day when the notice is given or deemed to be given; and
    \item
      the day for which it is given or on which it is to take effect
    \end{itemize}
  \item[Companies Acts] means the Companies Acts (as defined in section
    2 of the Companies Act 2006) insofar as they apply to the association
  \item[The Directors] means the directors of the association.
  \item[Document] includes, unless otherwise specified, any document sent or supplied in electronic form
  \item[Electronic Form] has the meaning given in section 1168 of the Companies Act 2006
  \item[Officers] includes the directors and the secretary (if any)
  \item[Secretary] means any person appointed to perform the duties of
    the secretary of the association.
  \end{description}
  Words importing one gender shall include all genders, and the singular
  includes the plural and vice versa.

  Unless the context otherwise requires, words or expressions
  contained in the articles have the same meaning as in the Companies
  Acts as in force on the date when these Articles become binding
  on the association.

  Apart from the exception mentioned in the previous paragraph a
  reference to an Act of Parliament includes any statutory
  modification or re-enactment of it for the time being in force.

\section{Liability of Members}

\item
  The liability of the members is limited to £1, being the amount that
  each member undertakes to contribute to the assets of the association
  in the event of its being wound up while he, she or it is a member or
  within one year after he, she or it ceases to be a member, for:
  \begin{enumerate}
  \item
    payment of the association's debts and liabilities incurred before he,
    she or it ceases to be a member;
  \item
    payment of the costs, charges and expenses of winding up; and
  \item
    adjustment of the rights of the contributories among themselves.
  \end{enumerate}

\section{Objects}

\item
  The objects of the association are:
  \begin{enumerate}
  \item
    to promote and encourage technical, scientific, and artistic skills through social collaboration and education; and
  \item
    to provide and maintain shared community workspace and equipment in Greater London.
  \end{enumerate}

\section{Powers}

\item
  The association has power to do anything which is calculated to further
  its Objects or is conducive or incidental to doing so. In
  particular, the association has power:
  \begin{enumerate}
  \item
    to raise funds. 
  \item
    to buy, take on lease or in exchange, hire or otherwise acquire any
    property and to maintain and equip it for use;
  \item
    to sell, lease or otherwise dispose of all or any part of the
    property belonging to the association.  
  \item
    to borrow money and to charge the whole or any part of the property
    belonging to the association as security for repayment of the money
    borrowed or as security for a grant or the discharge of an
    obligation.
  \item
    to acquire, merge with or to enter into any partnership or joint
    venture arrangement with any other association;
  \item
    to set aside income as a reserve against future expenditure but
    only in accordance with a written policy about reserves;
  \item
    to employ and remunerate such staff as are necessary for carrying
    out the work of the association. The association may employ or remunerate a
    director only to the extent it is permitted to do so by article \ref{cls:income-property}
    and provided it complies with the conditions in that article;
  \item
    to:
    \begin{enumerate}
    \item
      deposit or invest funds; and
    \item
      employ a professional fund-manager.
    \end{enumerate}
  \item
    to provide indemnity insurance for the directors.
  \item
    to pay out of the funds of the association the costs of forming and
    registering the association;
  \end{enumerate}

\section{Application of Income and Property}

\item
  \label{cls:income-property}
  \begin{enumerate}
  \item
    The income and property of the association shall be applied solely
    towards the promotion of the Objects.
  \item
    \label{scls:director-reimburse}
    \begin{enumerate}
    \item
      A director is entitled to be reimbursed from the property of the
      association or may pay out of such property reasonable expenses
      properly incurred by him or her when acting on behalf of the
      association.
    \item
      A director may receive an indemnity from the association in the
      circumstances specified in article \ref{cls:directors-indemnity}.
    \end{enumerate}
  \item
    None of the income or property of the association may be paid or transferred,
    directly or indirectly, by way of dividend, bonus, or otherwise by way of
    profit to any member of the association. This does not prevent a member who
    is not also a director receiving:
    \begin{enumerate}
    \item
      a benefit from the association in the capacity of a beneficiary of the
      association;
    \item
      reasonable and proper remuneration for any goods or services
      supplied to the association.
    \end{enumerate}
  \end{enumerate}

\section{Members}

\item
  \label{cls:subscribers}
  \begin{enumerate}
  \item
    The subscribers to the memorandum are the first members of the association.
  \item
    Membership is open to other individuals or organisations who:
    \begin{enumerate}
    \item
      apply to the association in the form required by the directors;
    \item
      supply a valid e-mail address; and
    \item
      are approved by the directors
    \end{enumerate}
  \item

    \begin{enumerate}
    \item
      The directors may only refuse an application for membership if,
      acting reasonably and properly, they consider it to be in the best
      interests of the association to refuse the application.
    \item
      The directors must inform the applicant in writing of the reasons
      for the refusal within twenty-one days of the decision.
    \item
      The directors must consider any written representations the
      applicant may make about the decision. The directors' decision
      following any written representations must be notified to the
      applicant in writing but shall be final.
    \end{enumerate}
  \item
    Membership is not transferable.
  \item
    The directors must keep a register of names and addresses of the
    members.
  \end{enumerate}

\section{Termination of Membership}

\item
  Membership is terminated if:
  \begin{enumerate}
  \item
    the member dies or, if it is an organisation, ceases to exist;
  \item
    the member resigns by written notice to the association unless, after
    the resignation, there would be fewer than three members;
  \item
    any sum due from the member to the association is not paid in full
    within two weeks of it falling due;
  \item
    the member is removed from membership by a resolution of the
    directors that it is in the best interests of the association that his
    or her or its membership is terminated. A resolution to remove a
    member from membership may only be passed if:
    \begin{enumerate}
    \item
      the member has been given at least twenty-one days' notice in
      writing of the meeting of the directors at which the resolution
      will be proposed and the reasons why it is to be proposed;
    \item
      the member or, at the option of the member, the member's
      representative (who need not be a member of the association) has been
      allowed to make representations to the meeting.
    \end{enumerate}
  \end{enumerate}

\section{General Meetings}
\item
    The Board may convene a General Meeting whenever they think fit. 

\item 
    General Meetings shall also be convened on such requisition, or in default may be convened by such
    requisitionists, as provided by section 304 of the Companies Act 2006.

\section{Notice of General Meetings}

\item
  \begin{enumerate}
  \item
    The minimum periods of notice required to hold a general meeting of
    the association are:
    \begin{enumerate}
    \item
      twenty-one clear days for an annual general meeting or a general
      meeting called for the passing of a special resolution;
    \item
      fourteen clear days for all other general meetings.
    \end{enumerate}
  \item
    A general meeting may be called by shorter notice if it is so
    agreed by a majority in number of members having a right to attend
    and vote at the meeting, being a majority who together hold not
    less than 90 percent of the total voting rights.
  \item
    The notice must specify the date time and place of the meeting and
    the general nature of the business to be transacted. If the meeting
    is to be an annual general meeting, the notice must say so. The
    notice must also contain a statement setting out the right of
    members to appoint a proxy under section 324 of the Companies Act
    2006 and article \ref{proxies}.
  \item
    The notice must be given to all the members and to the directors
    and auditors.
  \end{enumerate}

\item
  The proceedings at a meeting shall not be invalidated because a
  person who was entitled to receive notice of the meeting did not
  receive it because of an accidental omission by the association.

\section{Proceedings at General Meetings}

\item
  \begin{enumerate}
  \item
    No business shall be transacted at any general meeting unless a
    quorum is present.
  \item
    A quorum is 10 members present in person or by proxy and entitled to vote upon
      the business to be conducted at the meeting.
  \item
    The authorised representative of a member organisation shall be
    counted in the quorum.
  \end{enumerate}

\item
  \begin{enumerate}
  \item
    If:
    \begin{enumerate}
    \item
      a quorum is not present within half an hour from the time appointed
      for the meeting; or
    \item
      during a meeting a quorum ceases to be present; the meeting shall
      be adjourned to such time and place as the directors shall
      determine.
    \end{enumerate}
  \item
    The directors must reconvene the meeting and must give at least
    seven clear days' notice of the reconvened meeting stating the
    date, time and place of the meeting.
  \item
    If no quorum is present at the reconvened meeting within fifteen
    minutes of the time specified for the start of the meeting the
    members present in person or by proxy at that time shall constitute
    the quorum for that meeting.
  \end{enumerate}
\item
  

  \begin{enumerate}
  \item
    General meetings shall be chaired by the person who has been
    appointed to chair meetings of the directors.
  \item
    If there is no such person or he or she is not present within
    fifteen minutes of the time appointed for the meeting a director
    nominated by the directors shall chair the meeting.
  \item
    If there is only one director present and willing to act, he or she
    shall chair the meeting.
  \item
    If no director is present and willing to chair the meeting within
    fifteen minutes after the time appointed for holding it, the
    members present in person or by proxy and entitled to vote must
    choose one of their number to chair the meeting.
  \end{enumerate}
\item
  

  \begin{enumerate}
  \item
    The members present in person or by proxy at a meeting may resolve
    by ordinary resolution that the meeting shall be adjourned.
  \item
    The person who is chairing the meeting must decide the date, time
    and place at which the meeting is to be reconvened unless those
    details are specified in the resolution.
  \item
    No business shall be conducted at a reconvened meeting unless it
    could properly have been conducted at the meeting had the
    adjournment not taken place.
  \item
    If a meeting is adjourned by a resolution of the members for more
    than seven days, at least seven clear days' notice shall be given
    of the reconvened meeting stating the date, time and place of the
    meeting.
  \end{enumerate}
\item
  

  \begin{enumerate}
  \item
    Any vote at a meeting shall be decided by a show of hands unless
    before, or on the declaration of the result of, the show of hands a
    poll is demanded:
    \begin{enumerate}
    \item
      by the person chairing the meeting; or
    \item
      by at least two members present in person or by proxy and having
      the right to vote at the meeting; or
    \item
      by a member or members present in person or by proxy representing
      not less than one-tenth of the total voting rights of all the
      members having the right to vote at the meeting.
    \end{enumerate}
  \item
    

    \begin{enumerate}
    \item
      The declaration by the person who is chairing the meeting of the
      result of a vote shall be conclusive unless a poll is demanded.
    \item
      The result of the vote must be recorded in the minutes of the
      association but the number or proportion of votes cast need not be
      recorded.
    \end{enumerate}
  \item
    

    \begin{enumerate}
    \item
      A demand for a poll may be withdrawn, before the poll is taken, but
      only with the consent of the person who is chairing the meeting.
    \item
      If the demand for a poll is withdrawn the demand shall not
      invalidate the result of a show of hands declared before the demand
      was made.
    \end{enumerate}
  \item
    

    \begin{enumerate}
    \item
      A poll must be taken as the person who is chairing the meeting
      directs, who may appoint scrutineers (who need not be members) and
      who may fix a time and place for declaring the results of the poll.
    \item
      The result of the poll shall be deemed to be the resolution of the
      meeting at which the poll is demanded.
    \end{enumerate}
  \item
    

    \begin{enumerate}
    \item
      A poll demanded on the election of a person to chair a meeting or
      on a question of adjournment must be taken immediately.
    \item
      A poll demanded on any other question must be taken either
      immediately or at such time and place as the person who is chairing
      the meeting directs.
    \item
      The poll must be taken within thirty days after it has been
      demanded.
    \item
      If the poll is not taken immediately at least seven clear days'
      notice shall be given specifying the time and place at which the
      poll is to be taken.
    \item
      If a poll is demanded the meeting may continue to deal with any
      other business that may be conducted at the meeting.
    \end{enumerate}
  \end{enumerate}

\section{Content of Proxy Notices}

\item \label{proxies}
  \begin{enumerate}
  \item
    Proxies may only validly be appointed by a notice in writing (a
    ``proxy notice'') which -
    \begin{enumerate}
    \item
      states the name and address of the member appointing the proxy;
    \item
      identifies the person appointed to be that member's proxy and the
      general meeting in relation to which that person is appointed;
    \item
      is signed by or on behalf of the member appointing the proxy, or is
      authenticated in such manner as the directors may determine; and
    \item
      is delivered to the association in accordance with the articles and any
      instructions contained in the notice of the general meeting to
      which they relate.
    \end{enumerate}
  \item
    The association may require proxy notices to be delivered in a
    particular form, and may specify different forms for different
    purposes.
  \item
    Proxy notices may specify how the proxy appointed under them is to
    vote (or that the proxy is to abstain from voting) on one or more
    resolutions.
  \item
    Unless a proxy notice indicates otherwise, it must be treated as -
    \begin{enumerate}
    \item
      allowing the person appointed under it as a proxy discretion as to
      how to vote on any ancillary or procedural resolutions put to the
      meeting; and
    \item
      appointing that person as a proxy in relation to any adjournment of
      the general meeting to which it relates as well as the meeting
      itself.
    \end{enumerate}
  \end{enumerate}

\section{Delivery of Proxy Notices}

\item
    \begin{enumerate}
        \item
        A person who is entitled to attend, speak or vote
        (either on a show of hands or on a poll) at a general meeting
        remains so entitled in respect of that meeting or any adjournment
        of it, even though a valid proxy notice has been delivered to the
        association by or on behalf of that person.
        \item
        An appointment under a
        proxy notice may be revoked by delivering to the association a notice
        in writing given by or on behalf of the person by whom or on whose
        behalf the proxy notice was given. 
        \item
        A notice revoking a proxy
        appointment only takes effect if it is delivered before the start
        of the meeting or adjourned meeting to which it relates 
        \item
        If a
        proxy notice is not executed by the person appointing the proxy, it
        must be accompanied by written evidence of the authority of the
        person who executed it to execute it on the appointor's behalf.
    \end{enumerate}

\section{Written Resolutions}

\item
  \begin{enumerate}
  \item
    A resolution in writing agreed by a simple majority (or in the case
    of a special resolution by a majority of not less than 75\%) of the
    members who would have been entitled to vote upon it had it been
    proposed at a general meeting shall be effective provided that:
    \begin{enumerate}
    \item
      a copy of the proposed resolution has been sent to every eligible
      member;
    \item
      a simple majority (or in the case of a special resolution a
      majority of not less than 75\%) of members has signified its
      agreement to the resolution; and
    \item
      it is contained in an authenticated document which has been
      received at the registered office within the period of 28 days
      beginning with the circulation date.
    \end{enumerate}
  \item
    A resolution in writing may comprise several copies to which one or
    more members have signified their agreement.
  \item
    In the case of a member that is an organisation, its authorised
    representative may signify its agreement.
  \end{enumerate}

\section{Votes of Members}

\item
  Every member, whether an individual or an organisation, shall have one vote.

\item
  Any objection to the qualification of any voter must be raised at
  the meeting at which the vote is tendered and the decision of the
  person who is chairing the meeting shall be final.
\item
  \begin{enumerate}
  \item
    Any organisation that is a member of the association may nominate any
    person to act as its representative at any meeting of the association.
  \item
    The organisation must give written notice to the association of the
    name of its representative. The representative shall not be
    entitled to represent the organisation at any meeting unless the
    notice has been received by the association. The representative may
    continue to represent the organisation until written notice to the
    contrary is received by the association.
  \item
    Any notice given to the association will be conclusive evidence that
    the representative is entitled to represent the organisation or
    that his or her authority has been revoked. The association shall not
    be required to consider whether the representative has been
    properly appointed by the organisation.
  \end{enumerate}

\section{Directors}

\item
  \begin{enumerate}
  \item
    A director must be a natural person aged 16 years or older.
    No one may be appointed a director if he or she would be
    disqualified from acting under the provisions of article
    \ref{director-cease}.
  \item
    The number of directors shall be not less than three, nor more than nine.

  \item
    The first directors shall be those persons notified to Companies
    House as the first directors of the association.

  \item
    A director may not appoint an alternate director or anyone to act
    on his or her behalf at meetings of the directors.
  \end{enumerate}

\section{Powers of Directors}

\item
  \begin{enumerate}
  \item
    The directors shall manage the business of the association and may
    exercise all the powers of the association unless they are subject to
    any restrictions imposed by the Companies Acts, the articles or any
    special resolution.
  \item
    No alteration of the articles or any special resolution shall have
    retrospective effect to invalidate any prior act of the directors.
  \item
    Any meeting of directors at which a quorum is present at the time
    the relevant decision is made may exercise all the powers
    exercisable by the directors.
  \end{enumerate}

\section{Directors Elections}
\item
  \begin{enumerate}
    \item Directors elections must be called by the Board
      \begin{enumerate}
        \item in the case of the first election: before the end of the calendar year following
                the year in which the association was created.
        \item in the case of subsequent elections: before the end of the calendar year following
                the year in which the last election was held.
      \end{enumerate}
    \item
      The directors election must be carried out by electronically polling the membership as follows:
      \begin{enumerate}
        \item The Meek STV voting method with a "no further places" candidate must be used.
        \item All members of the association at the time the election commences are eligible to vote.
        \item The voting period of the election must be no less than 14 days and no more than 28 days.
        \item The quorum of the election must be at least 10\% of eligible voters.
      \end{enumerate}
    \item Directors standing for re-election will remain directors until the successful conclusion
          of the election.
    \item Notice of the election must be given at the time the election starts to all members eligible
          to vote who have registered an electronic mail address with the association.
   \end{enumerate}

\section{Retirement of Directors}
\item
    \begin{enumerate}
        \item At the first directors election, all the directors must retire from office unless the election
              fails to elect sufficient directors to hold a quorate meeting of the directors.
        \item At each subsequent directors election, one-third of the directors or, if their number is
              not three or a multiple of three, the number nearest to one-third, must retire from office.
              If there is only one director, they must retire.
        \item The directors to retire by rotation shall be those who have been longest in office since their
              last appointment. If any directors became or were appointed directors
              on the same day, those to retire shall (unless they otherwise agree among themselves)
              be determined by lot.
        \item Retiring directors may stand for re-election.
    \end{enumerate}

\section{Appointment of Directors}

\item
  The association may by ordinary resolution:
  \begin{enumerate}
  \item
    appoint a person who is willing to act to be a director; and
  \item
    determine the rotation in which any additional directors are to
    retire.
  \end{enumerate}

\item
  No person other than a director retiring by rotation may be
  appointed a director at any general meeting unless:
  \begin{enumerate}
  \item
    he or she is recommended for election by the directors; or
  \item
    not less than fourteen nor more than thirty-five clear days before
    the date of the meeting, the association is given a notice that:
    \begin{enumerate}
    \item
      is signed by a member entitled to vote at the meeting;
    \item
      states the member's intention to propose the appointment of a
      person as a director;
    \item
      contains the details that, if the person were to be appointed, the
      association would have to file at Companies House; and
    \item
      is signed by the person who is to be proposed to show his or her
      willingness to be appointed.
    \end{enumerate}
  \end{enumerate}
\item
  All members who are entitled to receive notice of a general meeting
  must be given not less than seven nor more than twenty-eight clear
  days' notice of any resolution to be put to the meeting to appoint
  a director other than a director who is to retire by rotation.

\item
  \begin{enumerate}
  \item
    The directors may appoint a person who is willing to act to be a
    director.
  \item
    A director appointed by a resolution of the other directors must
    retire at the next annual general meeting and must not be taken
    into account in determining the directors who are to retire by
    rotation.
  \end{enumerate}

\section{Disqualification and Removal of Directors}

\item \label{director-cease}
  A director shall cease to hold office if he or she:
  \begin{enumerate}
    \item
    ceases to be a director by virtue of any provision in the
    Companies Acts or is prohibited by law from being a director; 
    \item
    ceases to be a member of the association; 
    \item
    becomes incapable by reason of mental disorder, illness
    or injury of managing and administering his or her own affairs;
    \item 
    resigns as a director by notice to the association (but only if at
    least two directors will remain in office when the notice of
    resignation is to take effect); or 
    \item 
    is absent without the
    permission of the directors from all their meetings held within a
    period of six consecutive months and the directors resolve that his
    or her office be vacated.
  \end{enumerate}

\section{Remuneration of Directors}

\item
  The directors must not be paid any remuneration unless it is
  authorised by article \ref{cls:income-property}.

\section{Proceedings of Directors}

\item
  \begin{enumerate}
    \item
    The Directors may from time to time specify a web based system (``the
    Governance System'') for recording and managing their decision making process.
    \item
    The Governance System must:
        \begin{enumerate}
            \item
             permit any director to create a proposal that is then available for viewing (on
             presentation of suitable credentials) by all directors and any member of the
             Association;
            \item
            after a proposal has been created, send it promptly to each director's email address;
            \item
             maintain one or more rules as to the period (``the voting period'') within which any
             proposal must be accepted or rejected, which may vary from proposal to proposal;
            \item
             during the voting period, permit any director to indicate whether they accept or reject
             the proposal;
            \item
             record the votes cast by each director;
            \item
             record that a proposal has been ``passed'' if the requisite number of directors have
             indicated their acceptance of the proposal via the governance system;
            \item
             permit the recording of minutes; and
            \item
             maintain a current contact email address for each director (``the director's email
             address'').
        \end{enumerate}
    \item
    Any proposal that has been recorded as ``passed'' by the Governance System shall
    be treated as a decision of the directors.
    \item
    The Governance System to be used on the adoption of these articles shall be that
    supplied by One Click Orgs and accessed on the World Wide Web at the address
    http://london-gov.hackspace.org.uk
  \end{enumerate}

\item
  \begin{enumerate}
    \item
    Decisions of the directors may be made either:
    \begin{enumerate}
        \item
        by the use of the Governance System; or
        \item
        by a written resolution in accordance with the procedure described below.
    \end{enumerate}
    \item
    All directors' decisions must be made in one of these two ways, and a decision
    made in any other way will be ineffective.
    \item
    A majority of the directors must accept a proposal made using the
    Governance System or a written resolution before it is passed.
  \end{enumerate}

\item
    A written resolution may be passed if:
    \begin{enumerate}
        \item
        the text of the resolution is circulated to all directors in writing; and
        \item
        the requisite number of directors have given to all the other directors in writing their
        agreement to the text of the resolution.
    \end{enumerate}

\item
    \begin{enumerate}
    \item
        The directors may delegate any of their powers or functions concerning the day
        to day management of the affairs of the Company to any person or persons as they may
        from time to time select.
    \item
        All acts done by a director shall, even if it is afterwards discovered that there was
        a defect in their appointment or that they were disqualified from holding office or had
        vacated office, be as valid as if such person had been duly appointed and was qualified
        and had continued to be a director.
    \item
        Where on any particular occasion more than 2 directors, in the course of their
        office as directors, discuss the affairs of the Company by any means, whether in person
        or otherwise, a minute shall be made and recorded indicating:
            \begin{enumerate}
                \item
                    when the discussion took place;
                \item
                    the identities of the directors involved; and
                \item
                    the substance of the discussion.
            \end{enumerate}
    \item
        Minutes of directors meetings:
            \begin{enumerate}
                \item
                    may be held in electronic form, and in particular using the Governance System;
                \item
                    must be held for ten years from the date of the discussion in accordance with section
                    248 of the Companies Act 2006.
            \end{enumerate}
    \end{enumerate}

\section{Delegation}

\item
  \begin{enumerate}
  \item
    The directors may delegate any of their powers or functions to a
    committee of two or more directors but the terms of any delegation
    must be recorded in the minute book.
  \item
    The directors may impose conditions when delegating, including the
    conditions that:
    \begin{enumerate}
    \item
      the relevant powers are to be exercised exclusively by the
      committee to whom they delegate;
    \item
      no expenditure may be incurred on behalf of the association except in
      accordance with a budget previously agreed with the directors.
    \end{enumerate}
  \item
    The directors may revoke or alter a delegation.
  \item
    All acts and proceedings of any committees must be fully and
    promptly reported to the directors.
  \end{enumerate}

\section{Declaration of Directors' Interests}

\item \label{directors-interests}
  A director must declare the nature and extent of any interest,
  direct or indirect, which he or she has in a proposed transaction
  or arrangement with the association or in any transaction or
  arrangement entered into by the association which has not previously
  been declared. A director must absent himself or herself from any
  discussions of the directors in which it is possible that a
  conflict will arise between his or her duty to act solely in the
      interests of the association and any personal interest (including but
  not limited to any personal financial interest).

\section{Conflicts of Interests}

\item 
  \begin{enumerate}
  \item
    If a conflict of interests arises for a director because of a duty
    of loyalty owed to another organisation or person and the conflict
    is not authorised by virtue of any other provision in the articles,
    the unconflicted directors may authorise such a conflict of
    interests where the following conditions apply:
    \begin{enumerate}
    \item
      the conflicted director is absent from the part of the meeting at
      which there is discussion of any arrangement or transaction
      affecting that other organisation or person;
    \item
      the conflicted director does not vote on any such matter and is not
      to be counted when considering whether a quorum of directors is
      present at the meeting; and
    \item
      the unconflicted directors consider it is in the interests of the
      association to authorise the conflict of interests in the circumstances
      applying.
    \end{enumerate}
  \item \label{conflict-interest-loyalty}
    In this article a conflict of interests arising because of a duty
    of loyalty owed to another organisation or person only refers to
    such a conflict which does not involve a direct or indirect benefit
    of any nature to a director.
  \end{enumerate}

\section{Validity of Directors' Decisions}

\item
  \begin{enumerate}
  \item \label{directors-validity}
    Subject to article \ref{directors-validity-void}, all acts done by a meeting of directors,
    or of a committee of directors, shall be valid notwithstanding the
    participation in any vote of a director:
    \begin{enumerate}
    \item
      who was disqualified from holding office;
    \item
      who had previously retired or who had been obliged by the
      constitution to vacate office;
    \item
      who was not entitled to vote on the matter, whether by reason of a
      conflict of interests or otherwise; if without:
    \item
      the vote of that director; and
    \item
      that director being counted in the quorum; the decision has been
      made by a majority of the directors at a quorate meeting.
    \end{enumerate}
  \item \label{directors-validity-void}
    Article \ref{directors-validity} does not permit a director to
    keep any benefit that may be conferred upon him or her by a
    resolution of the directors or of a committee of directors if, but
    for article \ref{directors-validity}, the resolution would have been void, or if the
    director has not complied with article \ref{directors-interests}.
  \end{enumerate}

\section{Minutes}

\item
The directors must keep minutes of all:
    \begin{enumerate}
        \item
            appointments of officers made by the directors;
        \item
            proceedings at meetings of the association;
        \item
            meetings of the directors and committees of directors including:
            \begin{enumerate}
                \item
                    the names of the directors present at the meeting;
                \item
                    the decisions made at the meetings; and
                \item
                    where appropriate the reasons for the decisions.
            \end{enumerate}
    \end{enumerate}


\section{Accounts}

\item
  \begin{enumerate}
  \item
    The directors must prepare for each financial year accounts as
    required by the Companies Acts. The accounts must be prepared to
    show a true and fair view and follow accounting standards issued or
    adopted by the Accounting Standards Board or its successors and
    adhere to the recommendations of applicable Statements of
    Recommended Practice.
  \item
    The directors must keep accounting records as required by the
    Companies Acts.
  \end{enumerate}


\section{Means of Communication to be Used}

\item
  \begin{enumerate}
  \item
    Subject to the articles, anything sent or supplied by or to the
    association under the articles may be sent or supplied in any way in
    which the Companies Act 2006 provides for documents or information
    which are authorised or required by any provision of that Act to be
    sent or supplied by or to the association.
  \item
    Subject to the articles, any notice or document to be sent or
    supplied to a director in connection with the taking of decisions
    by directors may also be sent or supplied by the means by which
    that director has asked to be sent or supplied with such notices or
    documents for the time being.
  \end{enumerate}
\item
  Any notice to be given to or by any person pursuant to the
  articles:
  \begin{enumerate}
  \item
    must be in writing; or
  \item
    must be given in electronic form.
  \end{enumerate}

\item
  \begin{enumerate}
  \item
    The association may give any notice to a member either:
    \begin{enumerate}
    \item
      personally; or
    \item
      by sending it by post in a prepaid envelope addressed to the member
      at his or her address; or
    \item
      by leaving it at the address of the member; or
    \item
      by giving it in electronic form to the member's address.
    \item
      by placing the notice on a website and providing the person with a
      notification in writing or in electronic form of the presence of
      the notice on the website. The notification must state that it
      concerns a notice of a company meeting and must specify the place
      date and time of the meeting.
    \end{enumerate}
  \item
    A member who does not register an address with the association or who
    registers only a postal address that is not within the United
    Kingdom shall not be entitled to receive any notice from the
    association.
  \end{enumerate}
\item
  A member present in person at any meeting of the association shall be
  deemed to have received notice of the meeting and of the purposes
  for which it was called.

\item
  \begin{enumerate}
  \item
    Proof that an envelope containing a notice was properly addressed,
    prepaid and posted shall be conclusive evidence that the notice was
    given.
  \item
    Proof that an electronic form of notice was given shall be
    conclusive where the company can demonstrate that it was
    accepted by an electronic mail host for that address.
  \item
    In accordance with section 1147 of the Companies Act 2006, notice
    shall be deemed to be given:
    \begin{enumerate}
    \item
      48 hours after the envelope containing it was posted; or
    \item
      in the case of an electronic form of communication, 48 hours after
      it was sent.
    \end{enumerate}
  \end{enumerate}

\section{Indemnity}

\item \label{cls:directors-indemnity}
  \begin{enumerate}
  \item
    The association may indemnify any director against any
    liability incurred by him or her or it in that capacity, to the
    extent permitted by sections 232 to 234 of the Companies Act 2006.
  \item
    In this article a ``relevant director'' means any director or
    former director of the association.
  \end{enumerate}

\section{Rules}

\item
  \begin{enumerate}
  \item
    The directors may from time to time make such reasonable and proper
    rules or bye laws as they may deem necessary or expedient for the
    proper conduct and management of the association.
  \item
    The bye laws may regulate the following matters but are not
    restricted to them:
    \begin{enumerate}
    \item
      the admission of members of the association (including the admission of
      organisations to membership) and the rights and privileges of such
      members, and the entrance fees, subscriptions and other fees or
      payments to be made by members;
    \item
      the conduct of members of the association in relation to one another,
      and to the association's employees and volunteers;
    \item
      the setting aside of the whole or any part or parts of the
      association's premises at any particular time or times or for any
      particular purpose or purposes;
    \item
      the procedure at general meetings and meetings of the directors in
      so far as such procedure is not regulated by the Companies Acts or
      by the articles;
    \item
      generally, all such matters as are commonly the subject matter of
      company rules.
    \end{enumerate}
  \item
    The association in general meeting has the power to alter, add to or
    repeal the rules or bye laws.
  \item
    The directors must adopt such means as they think sufficient to
    bring the rules and bye laws to the notice of members of the
    association.
  \item
    The rules or bye laws shall be binding on all members of the
    association. No rule or bye law shall be inconsistent with, or shall
    affect or repeal anything contained in, the articles.
  \end{enumerate}

\section{Disputes}

\item
    If a dispute arises between members of the company about the validity
    or propriety of anything done by the members of the company under these
    articles, and the dispute cannot be resolved by agreement, the parties
    to the dispute must first try in good faith to settle the dispute by
    mediation before resorting to litigation.

\section{Dissolution}

\item
  \begin{enumerate}
  \item \label{dissolve-transfer}
    The members of the association may at any time before, and in
    expectation of, its dissolution resolve that any net assets of the
    association after all its debts and liabilities have been paid, or
    provision has been made for them, shall on or before the
    dissolution of the association be applied or transferred in any of the
    following ways:
    \begin{enumerate}
    \item
      directly for the Objects; or
    \item
      by transfer to any association for purposes similar to the
      Objects; or
    \item
      to any association for use for particular purposes that
      fall within the Objects.
    \end{enumerate}
  \item
    Subject to any such resolution of the members of the association, the
    directors of the association may at any time before and in expectation
    of its dissolution resolve that any net assets of the association after
    all its debts and liabilities have been paid, or provision made for
    them, shall on or before dissolution of the association be applied or
    transferred:
    \begin{enumerate}
    \item
      directly for the Objects; or
    \item
      by transfer to any association for purposes similar to the
      Objects; or
    \item
      to any association for use for particular purposes that
      fall within the Objects.
    \end{enumerate}
  \item
    In no circumstances shall the net assets of the association be paid to
    or distributed among the members of the association (except to a member
    that is itself a association) and if no resolution in accordance with
    article \ref{dissolve-transfer} is passed by the members or the directors the net
    assets of the association shall be applied for charitable purposes as
    directed by the Court.
  \end{enumerate}
\end{enumerate}
