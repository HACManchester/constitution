\section {Notes to the Financial Statements}
\begin{enumerate}
    \item {\bf Accounting policies}
    \begin{enumerate}
        \item {\bf Accounting convention} \\
                The financial statements are prepared under the 
            historical cost convention and in accordance with the Financial Reporting 
            Standard for Smaller Entities (effective April 2008).
        \item {\bf Compliance with accounting standards} \\
            The financial statements are prepared in accordance with applicable United
            Kingdom Accounting Standards (United Kingdom Generally Accepted Accounting
            Practice), which have been applied consistently throughout the period.
        \item {\bf Turnover} \\
            Turnover represents amounts receivable for services.
        \item {\bf Tangible fixed assets and depreciation} \\
            Tangible fixed assets are stated at cost less depreciation. Depreciation is
            provided at rates calculated to write off the cost less estimated residual
            value of each asset over its expected useful life, as follows: \\[6pt]
            \begin{tabular}{  l  p{6.5cm} }
                \toprule
                Fixtures \& Fittings & Over 5 years at 20\% per annum on a straight line basis \\
                Plant \& Machinery & Over 4 years at 25\% per annum on a straight line basis \\
                \bottomrule
            \end{tabular}
    \end{enumerate}
    \item {\bf Change of depreciation policy} \label{depreciation_change} \\ 
        The directors have chosen to alter the depreciation policy to be more in line
        with the expected useful life of equipment, fixtures, and fittings, based on
        observations made in the last year. Fixtures and Fittings are now depreciated
        over five years, and plant and machinery over four years on a straight-line
        basis.

        For the 2010/11 reporting period, these changes have resulted in the
        depreciation charge being reduced from £1,325 to £952, which represents a 4\%
        increase of the operating surplus.

        The depreciation expenditure for the 2009-2010 financial year has been restated
        according to the new depreciation policy. This has resulted in a change of £13
        to the depreciation line, or less than 1\% of the surplus for that year.

\end{enumerate}
